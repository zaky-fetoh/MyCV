\documentclass[a4paper,10pt]{article}
\usepackage{fontspec}
\usepackage{eurosym}
\usepackage[margin=0.8in]{geometry}
\usepackage{xunicode,xltxtra,url,parskip} 
\usepackage[usenames,dvipsnames]{xcolor}  
\usepackage{hyperref} 
\definecolor{linkcolour}{rgb}{0,0.2,0.6} 
\hypersetup{colorlinks,breaklinks,urlcolor=linkcolour,linkcolor=linkcolour} 
\usepackage{titlesec} 
\titleformat{\section}{\Large\scshape\raggedright}{}{0em}{}[\titlerule] 
\titlespacing{\section}{0pt}{3pt}{3pt} 
\usepackage{float}
\usepackage{placeins}
\usepackage{tabularray}
\usepackage{enumitem}
\usepackage{array, booktabs, longtable}
\newcolumntype{x}[1]{>{\raggedright}p{#1}}

\usepackage{fancyhdr}

\begin{document}
\pagestyle{fancy}
\fancyhead{} 
\renewcommand{\headrulewidth}{0pt}
\fancyfoot[R]{\textbf{Mahmoud Zaky Fetoh}}
\fancyfoot[C]{Page~\thepage~of~\pageref{LastPage}}

\font\fb=''[cmr10]'' 
\par{\centering{\Huge \textsc{Mahmoud Zaky Fetoh}}\bigskip\par} % Your name

\section{\textbf{Personal Data}}

\begin{tabular}{lp{4cm}l}
\textbf{\textsc{Name:}} 	Mahmoud Zaky Fetoh && \textbf{\textsc{Phone:}} +201022381474 \\

\textbf{\textsc{Address:}} Tanta, Egypt && \textbf{\textsc{Email:}}   \href{mailto:zaky.fetoh@gmail.com}{zaky.fetoh@gmail.com}  \\

\href{https://github.com/zaky-fetoh}{github.com/zaky-fetoh} && \href{https://www.linkedin.com/in/mahmoud-zaky-fetoh/}{linkedin.com/in/mahmoud-zaky-fetoh}  \\

\end{tabular}

\section{\textbf{Objective}}
~~~~~~I am a junior MERN stack engineer passionate about solving a real-world problem, working and collaborate with a great Engineers that I can learn from. I have strong knowledge of algorithms and writing efficient and high performing code. 

\section{\textbf{Education}}
\begin{tabular}{r|l}
    2020-Present  & \textbf{Faculty of Computers and Information, Menoufia university}, Egypt.\\ & M.Sc., doing computer vision research, Expected graduation  2023 \\
    &\\
    2015-2019 & \textbf{Faculty of Computers and Information, Menoufia university}, Egypt. \\ & B.Sc. Honors, I graduated with the Top, 1st, Grade, Excellent with honor GPA 3.5. \\
    \end{tabular}

\section{\textbf{Work Experience}}

\begin{tabular}{r|l}
2020-Present  & \textbf{Demonstrator} \\&\textbf{Faculty of Computers and Information, Menoufia university}.\\& Teaching Deep Learning, Computer Vision Courses and Speech recognition.
\\
\end{tabular}

\section{\textbf{Frameworks \& Technical Skills}}

\begin{tabular}{rl}	
    \textbf{Languages}:& Python, NodeJS, GoLang(learning)\\
    \textbf{Programming Paradigms}:&  Object-oriented programming (OOP) and Functional Programming (FP).\\
    \textbf{HTTP Servers Frameworks}:& ExpressJS, Flask.\\
    \textbf{Front-End Frameworks}:& ReactJS, Angular(learning).\\
    \textbf{Version Control Tools}:& Git, Github, Bitbucket\\
    \textbf{Client API}:& REST, GraphQL, gRPC, SocketIO\\
    \textbf{Databases}:& MONGODB, MySQL.\\
    \textbf{Deep Learning}:& PyTorch, Pandas, statsmodels, Prophet, Numpy, OpenCV.\\
    \textbf{Security Utilities}:& Joi, Jsonwebtoken (JWT).\\ 
    \textbf{Containerization Technology}:& Docker, Docker-Compose.\\
    \textbf{Message Broker}:& RabbitMQ.\\
    \textbf{Caching}:& Redis.\\
    \textbf{Testing}:& Jest.\\ 
\end{tabular}
%%END TABLE

 
\section{\textbf{Project Experience}}  

% \begin{small}
    % \begin{table}[h]
    \begin{longtable}{r p{16cm}}
        2023  & \textbf{Personal Blog Site}, [\href{https://github.com/zaky-fetoh/Personal-Blog-Site}{Link}].\\&
        This is a MERN stack project that manage personal blog and store it in MONGODB database. It performs caching for quary using redus. It's frontend is build using ReactJS and it state is managed using Redux-toolkit.
        \begin{itemize}
            \item \textit{built with: } NodeJS, ExpressJS,  ReactJS, Crypto, Redis, MongoDB, Docker-Compose, Redux-toolkit, formik, and yup.
        \end{itemize}\\

        2022  & \textbf{Clinic Management System}, [\href{https://github.com/zaky-fetoh/Clinic-Management-System}{Link}].\\&
        A backend System for managing clinics, it's department, medical stuff, and patients. This 	System provides a RESTfull, and GraphQL for clients. Complex mongodb aggregation pipeline 	is implemented for extracting high level info.
        \begin{itemize}
            \item \textit{built with: } NodeJS, Express, Mongoose, graphql, and Joi. 
        \end{itemize}\\


        2022  & \textbf{Image Manager}, [\href{https://github.com/zaky-fetoh/image-maneger-microservices-app}{Link}].  \\& Microservices application for image storage, stored in plain form or encrypted form, with it's meta data and automatic tagging using Deep learning techniques. \\&
        It consists of three microservices as follow:\\&
        \begin{itemize}
            \item \textbf{Image-Classification micro-service:}
            This micro-service is responsible for classifying images using ResNet18 other services can communicate with it using synchronous communication using by 
            requesting it RESTfull API or through asynchronous communication using RabbitMQ.
            \begin{itemize}
                \item \textit{built with: } Python, Pytorch, Flask, and pika.
            \end{itemize}

            \item \textbf{Image-Storage micro-service:}
            This micro-service is responsible for storing the images the users in either plain or encrypted, using AES, form other microservices can communicate with it using it's RESTfull API.
            \begin{itemize}
                \item \textit{built with: } NodeJS, Express, Multer, and crypto.
            \end{itemize}

            \item \textbf{Image-manager micro-service:}
            This micro-service is responsible for 1) storing the meta-data, such as Image's owner and the image Tags, automatically added by Image Classification micro-service, and the total number of views, 2) manage users, users credentials and authentication 3) communicate with Image-Classification microservice using RabbitMQ and With Image-Storage through it's RESTfull API.
            \begin{itemize}
                \item \textit{built with: }NodeJS, Express, Mongoose, JWT, bcrypt, amqplib, and Joi.
            \end{itemize}
        \end{itemize}\\

        2022  & \textbf{Notification Server}, [\href{https://github.com/zaky-fetoh/Learning-Backend-Development-Using-NodeJS/tree/main/Day10_NotificationSerWithCronJob}{Link}].\\&
        Periodically send a scheduled Notification using scheduled using CronJob. \\
    \end{longtable}
% \end{table}

% \end{small}
\label{LastPage}
\end{document}
