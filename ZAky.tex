\documentclass[a4paper,10pt]{article}
\usepackage{fontspec}
\usepackage{eurosym}
\usepackage[margin=0.8in]{geometry}
\usepackage{xunicode,xltxtra,url,parskip} 
\usepackage[usenames,dvipsnames]{xcolor}  
\usepackage{hyperref} 
\usepackage{float}
\usepackage{placeins}
\usepackage{tabularray}
\usepackage{enumitem}
\usepackage{array, booktabs, longtable}
\usepackage{titlesec} 
\usepackage{fancyhdr}
\definecolor{linkcolour}{rgb}{0,0.2,0.6} 
\hypersetup{colorlinks,breaklinks,urlcolor=linkcolour,linkcolor=linkcolour} 
\titleformat{\section}{\Large\scshape\raggedright}{}{0em}{}[\titlerule] 
\titlespacing{\section}{0pt}{3pt}{3pt} 
\newcolumntype{x}[1]{>{\raggedright}p{#1}}


\begin{document}
\pagestyle{fancy}
\fancyhead{} 
\renewcommand{\headrulewidth}{0pt}
\fancyfoot[R]{\textbf{Zaky Fetoh}}
% \fancyfoot[C]{[~Page~\thepage~of~\pageref{LastPage}~]}
\fancyfoot[C]{[~Page~\thepage~of~2~]}

\font\fb=''[cmr10]'' 
\par{\centering{\Huge \textsc{Mahmoud Zaky Fetoh}}\bigskip\par} % Your name

% \section{\textbf{Personal Data}}

\begin{tabular}{lp{4cm}l}
% \textbf{\textsc{Name:}} Mahmoud Zaky Fetoh && \textbf{\textsc{Phone:}} +201022381474 \\
% \textbf{\textsc{Address:}} Tanta, Egypt && \textbf{\textsc{Email:}}   \href{mailto:zaky.fetoh@gmail.com}{zaky.fetoh@gmail.com}  \\
\textbf{\textsc{Phone:}} +201022381474 && \href{https://github.com/zaky-fetoh}{github.com/zaky-fetoh} \\
\textbf{\textsc{Email:}}   \href{mailto:zaky.fetoh@gmail.com}{zaky.fetoh@gmail.com} &&\href{https://www.linkedin.com/in/mahmoud-zaky-fetoh/}{linkedin.com/in/mahmoud-zaky-fetoh}\\

\end{tabular}

\section{\textbf{Education}}
\begin{longtable}{r|l}
    2021 - 2023  & \textbf{Faculty of Computers and Information, Menoufia university}, Egypt.\\& 
    \textbf{M.Sc.} in Computer Vision, specialized in deep learning research, graduated in 2023. \\&
    $\circ$ Architected Convolutional Neural Networks (CNNs) for detecting COVID-19 in medical images. \\&
    $\circ$ Built training pipelines, validated model accuracy, and performed model comparison. \\&   
    $\circ$ Monitored experiments, documented results, and writing papers. \\
    &\\
    2015 - 2019 & \textbf{Faculty of Computers and Information, Menoufia university}, Egypt. \\& 
    \textbf{B.Sc.} Honors. I graduated with highest grade on my class with an Excellent honors GPA of 3.6. \\&
    $\circ$ Hired as a Teaching Assistant at the same institute. \\&
    $\circ$ My graduation project created an intrusion detection system for detecting various \\& network attacks in software-defined networks using deep learning CIC-IDS2017 is used a training \\& dataset\\
    &\\
    & \textbf{Professional Certificates.} \\
    & AWS Certified Solutions Architect - Associate [\href{https://cp.certmetrics.com/amazon/en/public/verify/credential/8ff7ef944a1c49cf873b541903d9cbc3}{Link}] \\
    & AWS Certified Machine Learning - Specialty [\href{https://cp.certmetrics.com/amazon/en/public/verify/credential/b5b06f594c0a45929e1971bf5215daf7}{Link}] \\
\end{longtable}

\section{\textbf{Publications}}  
    \begin{longtable}{r p{16cm}}
        2022  & \textbf{Multiscale aware classification of COVID-19 from Chest X-Ray using a spatially weighted atrous spatial pyramid pooling CNN} [\href{https://www.researchsquare.com/article/rs-3043485/v1}{Link}] [\href{https://github.com/zaky-fetoh/Multiscale-aware-classification-of-COVID-19-from-Chest-X-Ray-using-a-spatially-weighted-atrous}{github}].\\& \textbf{Mahmoud Z fetoh}, Khalid M. Amin, Ahmed M. Hamad \\&
        In this paper I propose, scale invariant CNN architecture for COVID-19 classification. Proposed model based on building a scale space in each layer using Atrous spatial pyramid pooling then selecting a correct space to operate at using spatial attention module.\\
        &\\ 
        2021  & \textbf{COVID-19 Detection Based on Chest X-Ray Image Classification using Tailored CNN Model}, [\href{https://scholar.google.com/scholar?cluster=12675761567539204826&hl=en&as_sdt=0,5}{Link}] [\href{https://github.com/zaky-fetoh/PhII-Cov_clf/tree/main}{github}].\\&
        \textbf{Mahmoud Z fetoh}, Khalid M. Amin, Ahmed M. Hamad \\&
        In this paper I propose a very light-weight model as a consequence of using spatial separable kernel and depth-wise separable kernels for COVID-19 classification.\\&
        \textbf{Published at}: IJCI.  
        \\
\end{longtable}

% \section{\textbf{Work Experience}}

% \begin{tabular}{r|l}
% 2020-Present  & \textbf{Demonstrator} \\&\textbf{Faculty of Computers and Information, Menoufia university}.\\& Teaching Deep Learning, Computer Vision Courses and Speech recognition.
% \\
% \end{tabular}
\section{\textbf{Work Experience}}
% May 2023 - Present
\begin{longtable}{r|l}
    Dec 2023 - Present& \textbf{ML Cloud Consultant} @
    \textbf{Bexprt}, UK. \\&
    Involved at architecting and implementing MLOps and ML projects for MENA \\& customers. \\&
    \\&
    \textit{- Job Role}:\\&
    $\circ$ Architecting scalable training and serving pipelines for offline/online ML models. \\&
    $\circ$ Managing AWS Cloud infrastructure across 7 AWS accounts for Backend,\\&~~Frontend, ML, and Data teams. \\&
    $\circ$ Architecting serverless cloud-native solutions and developing Infrastructure as \\&~~Code (IaC) for it. \\&
    $\circ$ Designing and implementing solutions for large-scale data processing and migration. \\&
    $\circ$ Creating and executing large-scale migration plans i.e) Moving from ECS to EKS. \\&
    $\circ$ Designing, developing, and maintaining scalable and reliable CI/CD pipelines. \\&
    $\circ$ Managing and maintaining compute clusters in AWS (ECS and EKS). \\&
    $\circ$ Conducting rigorous comparisons of 3rd-party tools, selecting and integrating \\& ~~them into the infrastructure. \\&
    $\circ$ Managing the monitoring solution and create tailored alerts. \\&
    $\circ$ Continuously creating and updating documentation. \\&
    \\&
    \textit{- Technologies}:\\&
    \textbf{Clould Provider}: AWS, Cloudflare. \\&
    \textbf{IaC}: Terraform, SAM, Serverless-Framework. \\&
    \textbf{CI/CD}: Github Actions, Atlantis, Terraform Cloud. \\&
    \textbf{ETL}: AWS Glue Crawler, Glue Database, Glue Job, S3, Firehose. \\&
    \textbf{MLOps}: SageMaker, bentoml, Prefect, MLflow, DVC. \\&
    \textbf{Micro-frontend}: Cloudfront, S3, Route53. \\&
    \textbf{Container Orchestrators}: ECS, EKS. \\&
    \textbf{Monitoring}: DataDog, NewRelic. \\&
    \\
    Jan 2021 - present & \textbf{Teaching \& Research Assistant} @ 
    \textbf{Menoufiya University}, Egypt. \\&
    $\circ$ Teaching courses on AI, deep learning, image processing, and computer vision. \\&
    $\circ$ Teaching alorithms, data structure, Operating systems courses. \\&
    \\
    Jan 2023 - Apr 2023 & \textbf{ML Engineer} @
    \textbf{Susoft}, Norway. \\& 
    Building microservice application for Training and deploying machine and deep learning \\& models for Sales forecasting. 
    Performing customer segmentation to direct marketing \\& campaigns. \\&
    \\&
    \textit{- Job Role}:\\&
    $\circ$ Validating if the problem can be resolved with AI or not \\&
    $\circ$ selecting the most accurate model. reproduce machine learning papers \\&
    $\circ$ Creating a training and serving ML pipeline for multiple time series 
    forecasting \\& ~~models, such as Prophet, Neural Prophet, and TFT. \\&
    $\circ$ Developing custom data pipelines for extracting, loading, and transforming  
    required \\& ~~data from MariaDB databases. \\&
    $\circ$ Evaluating and monitoring model performance. \\&
    $\circ$ Ensuring high availability of the serving models. \\&
    \\&
    \textit{- Technologies}:\\&
     \textbf{Model Training}: PyTorchForecasting, Pytorch, pandas, Prophet, NeuralProphet. \\&
     \textbf{Model Monitoring}: Weight and biases, Prometheus, Grafana.\\&
     \textbf{Model Serving}: torchScript, Docker, K8s. \\&
     \textbf{Model Registry}: Minio. \\&
    Asynchronous communication for issuing train request is done using RabbitMQ \\&
    Gateway and load balancing services performed using NodeJS.
\end{longtable}



\section{\textbf{Frameworks \& Technical Skills}}

\begin{longtable}{r p{16cm}}	
    \textbf{Programming Languages}:& Python, NodeJS, Bash\\
    \textbf{HTTP Servers Frameworks}:& ExpressJS, Flask.\\
    \textbf{MLOps}:& SageMaker, Prefect, bentoml, Mlflow, Hydra, DVC.\\
    \textbf{Data Science}:& PyTorch, Pandas, statsmodels, Prophet, Numpy, OpenCV, Plotly,\\
    \textbf{Monitoring}:& Prometheus, Grafana, Loki, ELK, kiali, jaeger, DataDog, NewRelic.\\ 
    \textbf{Infrastucture as Code}:& Terraform, Kustomize, Helm, Ansible.\\
    \textbf{Container Orchestrators}:& EKS, OKE, ECS, Docker-Compose. \\
    \textbf{Cloud Provider}:& AWS, OCI, Cloudflare. \\
    \textbf{Documentation}:& Swagger, \LaTeX.\\
\end{longtable}

\section{\textbf{Project Experience}}  
% \begin{small}
    % \begin{table}[h]
    \begin{longtable}{r p{16cm}}

        2021  & \textbf{MNIST latent space exploration}, [\href{https://github.com/zaky-fetoh/MNIST_latent_space_exploring}{Link}].\\&
        Projecting the MNIST database to 2D using Convolutional ConvAutoEncoder and SSIM using as loss function and calculating the entroy of the resultant space.
        \begin{itemize}
            \item \textit{tools used: } OpenCV, Pytorch, Numpy.
        \end{itemize}\\


        2021  & \textbf{Simple Stitching}, [\href{https://github.com/zaky-fetoh/stitching}{Link}].\\&
        simple stitching project which use SIFT as Keypoint for detecting and describtion then use RANSAC for Estimating the Homography and distance transform for Blending Simple project for image stitching
        \begin{itemize}
            \item \textit{tools used: } OpenCV, Numpy.
        \end{itemize}\\

        2021  & \textbf{MNIST Bayesian ConvAutoEncoder}, [\href{https://github.com/zaky-fetoh/MNIST-BayesianConvAutoEncoder}{Link}].\\&
        Implementing Bayesian ConvAutoEncoder trained using MNIST data set
        \begin{itemize}
            \item \textit{tools used: } OpenCV, Pytorch, Numpy.
        \end{itemize}\\

        2021  & \textbf{SPP-Net Multiscale Classification of Voc dataset}, [\href{https://github.com/zaky-fetoh/SPP-Net-implementation-for-multiscale-classification-of-VOC-dataset}{Link}].\\&
        implementaion of SPP-net paper using PyTorch. In this project a mutliscale and multilabel classifier is trained and evaluated Using Voc pascal dataset.
        \begin{itemize}
            \item \textit{tools used: } OpenCV, Pytorch, Numpy.
        \end{itemize}\\


        2021  & \textbf{Implemeting data preparation of RCNN paper}, [\href{https://github.com/zaky-fetoh/Implemeting_data_preparation_of_RCNN_paper}{Link}].\\&
        Implementing data preparation of Region-based Convolutional Neural Networks (R-CNN) paper
        \begin{itemize}
            \item \textit{tools used: } OpenCV, Pytorch, Numpy.
        \end{itemize}\\
    
    \end{longtable}

\label{LastPage}
\end{document}
