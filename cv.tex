
% from here: http://www.exljbris.com/fontin.html

\documentclass[a4paper,10pt]{article} % Default font size 
\usepackage{fontspec} % For loading fonts
\usepackage{eurosym}
\usepackage[margin=0.8in]{geometry}
%% IF YOU WANT TO COMPILE THIS TEX LOCALLY, COMMENT THE FOLLOWING LINES IF THE FONTS ARE NOT INSTALLED
% \defaultfontfeatures{Mapping=tex-text}
% \setmainfont{Fontin-Regular.otf}[
% BoldFont       = Fontin-Bold.otf ,
% ItalicFont     = Fontin-Italic.otf ,
% SmallCapsFont  = Fontin-SmallCaps.otf ]
%% UNTIL HERE 

\usepackage{xunicode,xltxtra,url,parskip} % Formatting packages
\usepackage[usenames,dvipsnames]{xcolor} % Required for specifying custom colors
% \usepackage[big]{layaureo} % Margin formatting of the A4 page, an alternative to layaureo can be 
% \usepackage{fullpage}
% \addtolength{\textwidth}{5cm}
% To reduce the height of the top margin uncomment: \addtolength{\voffset}{-1.3cm}
\usepackage{hyperref} % Required for adding links	and customizing them
\definecolor{linkcolour}{rgb}{0,0.2,0.6} % Link color
\hypersetup{colorlinks,breaklinks,urlcolor=linkcolour,linkcolor=linkcolour} % Set link colors throughout the document
\usepackage{titlesec} % Used to customize the \section command
\titleformat{\section}{\Large\scshape\raggedright}{}{0em}{}[\titlerule] % Text formatting of sections
\titlespacing{\section}{0pt}{3pt}{3pt} % Spacing around sections
\usepackage{float}
\usepackage{placeins}
\usepackage{tabularray}
\usepackage{enumitem}
\usepackage{array, booktabs, longtable}
\newcolumntype{x}[1]{>{\raggedright}p{#1}}

% \usepackage{etoolbox}
% \AtBeginEnvironment{longtable}{%
%     \setlist[itemize]{nosep,     % <-- new list setup
%                       topsep     = 0pt       ,
%                       partopsep  = 0pt       ,
%                       leftmargin = *         ,
%                       label      = $\bullet$ ,
%                       before     = \vspace{-\baselineskip},
%                       after      = \vspace{-0.5\baselineskip}
%                         }
%                            }% end of AtBeginEnvironment

%------------------
%------------------
%	MAIN DOCUMENT - PLEASE ENTER YOUR DATA HERE
%
%	Some basic LaTeX commands: 
%
%	A '%' marks a comment that will not be shown in the final PDF
%	If you want to remove content, just comment the line(s) with a leading '%'
%
%	When you open an environment with a '{' you need to close it with a corresponding '}'!
%
%	\textbf{TEXT} 						makes TEXT bold font
%	\emph{TEXT} 						makes TEXT italics font
%	\textsc{TEXT}						makes TEXT small caps font
%	\small \footnotesize \Huge ...		defines font sizes
%	\begin{tabular} ... \end{tabular} 	defines a table environment
%										inside a table environment a '&' sign splits between columns 
%										inside a table a '\\' marks the end of a row
%------------------
%------------------
\begin{document}
\pagestyle{empty} % Removes page numbering
\font\fb=''[cmr10]'' % Change the font of the \LaTeX command under the skills section

%------------------
%	NAME AND CONTACT INFORMATION
%------------------
\par{\centering{\Huge \textsc{Mahmoud Zaky Fetoh}}\bigskip\par} % Your name

\section{\textbf{Personal Data}}

\begin{tabular}{lp{4cm}l}
\textsc{Name:} 	Mahmoud Zaky Fetoh && \textsc{Phone:} +201022381474 \\

\textsc{Address:} Tanta && \textsc{email:}   \href{mailto:zaky.fetoh@gmail.com}{zaky.fetoh@gmail.com}  \\

\href{https://github.com/zaky-fetoh}{github.com/zaky-fetoh} && \href{https://www.linkedin.com/in/mahmoud-zaky-fetoh/}{linkedin.com/in/mahmoud-zaky-fetoh}  \\

\end{tabular}

%------------------
%	WORK EXPERIENCE 
%------------------


\section{\textbf{Objective}}
~~~~~~I am a junior MERN stack engineer passionate about solving a real-world problem, working and collaborate with a great Engineers that I can learn from. I have strong knowledge of algorithms and writing efficient and high performing code. 

\section{\textbf{Education}}
\begin{tabular}{r|l}
    2020-Present  & \textbf{Faculty of Computers and Information, Menoufia university}, Egypt.\\ & M.Sc., doing computer vision research, Expected graduation  2023 \\
    &\\
    2015-2019 & \textbf{Faculty of Computers and Information, Menoufia university}, Egypt. \\ & B.Sc. Honors, I graduated with the Top, 1st, Grade, Excellent with honor GPA 3.5. \\
    \end{tabular}

\section{\textbf{Work Experience}}

\begin{tabular}{r|l}
2020-Present  & \textbf{Demonstrator} \\&\textbf{Faculty of Computers and Information, Menoufia university}.\\& Teaching Deep Learning, Computer Vision Courses and Speech recognition.
\\
\end{tabular}

\section{\textbf{Frameworks \& Technical Skills}}

\begin{tabular}{rl}	
    \textbf{Languages}:& Python, NodeJS, GoLang(learning)\\
    \textbf{Programming Paradigms}:&  Object-oriented programming (OOP) and Functional Programming (FP).\\
    \textbf{HTTP Servers Frameworks}:& ExpressJS, Flask.\\
    \textbf{Front-End Frameworks} :& ReactJS, and Angular(learning).\\
    \textbf{Version Control Tools}:& Git, Github.\\
    \textbf{Client API}:& REST, GraphQL, gRPC,and SocketIO\\
    \textbf{Databases}:& MONGODB, MySQL.\\
    \textbf{Deep Learning}:& PyTorch, Numpy, Pandas, OpenCV.\\
    \textbf{Message Broker}:& RabbitMQ.\\
    \textbf{Security Utilities}:& Joi, Jsonwebtoken (JWT).\\ 
    \textbf{Containerization Technology}:& Docker, Docker-Compose.\\
    \textbf{Caching}:& Redis.\\
    \textbf{Testing}:& Jest.\\ 
\end{tabular}
%%END TABLE

 
\section{\textbf{Project Experience}}  

% \begin{small}
    % \begin{table}[h]
    \begin{longtable}{r p{16cm}}
        2023  & \textbf{Personal Blog Site}, [\href{https://github.com/zaky-fetoh/Personal-Blog-Site}{Link}].\\&
        This is a MERN stack project that manage personal blog and store it in MONGODB database. It performs caching for quary using redus. It's frontend is build using ReactJS and it state is managed using Redux-toolkit.
        \begin{itemize}
            \item \textit{built with: } NodeJS, ExpressJS,  ReactJS, Crypto, Redis, MongoDB, Docker-Compose, Redux-toolkit, formik, and yup.
        \end{itemize}\\

        2022  & \textbf{Clinic Management System}, [\href{https://github.com/zaky-fetoh/Clinic-Management-System}{Link}].\\&
        A backend System for managing clinics, it's department, medical stuff, and patients. This 	System provides a RESTfull, and GraphQL for clients. Complex mongodb aggregation pipeline 	is implemented for extracting high level info.
        \begin{itemize}
            \item \textit{built with: } NodeJS, Express, Mongoose, graphql, and Joi. 
        \end{itemize}\\

        
        2022  & \textbf{Image Manager}, [\href{https://github.com/zaky-fetoh/image-maneger-microservices-app}{Link}].  \\& Microservices application for image storage, stored in plain form or encrypted form, with it's meta data and automatic tagging using Deep learning techniques. \\&
        It consists of three microservices as follow:\\&
        \begin{itemize}
            \item \textbf{Image-Classification micro-service:}
            This micro-service is responsible for classifying images using ResNet18 other services can communicate with it using synchronous communication using by 
            requesting it RESTfull API or through asynchronous communication using RabbitMQ.
            \begin{itemize}
                \item \textit{built with: } Python, Pytorch, Flask, and pika.
            \end{itemize}

            \item \textbf{Image-Storage micro-service:}
            This micro-service is responsible for storing the images the users in either plain or encrypted, using AES, form other microservices can communicate with it using it's RESTfull API.
            \begin{itemize}
                \item \textit{built with: } NodeJS, Express, Multer, and crypto.
            \end{itemize}

            \item \textbf{Image-manager micro-service:}
            This micro-service is responsible for 1) storing the meta-data, such as Image's owner and the image Tags, automatically added by Image Classification micro-service, and the total number of views, 2) manage users, users credentials and authentication 3) communicate with Image-Classification microservice using RabbitMQ and With Image-Storage through it's RESTfull API.
            \begin{itemize}
                \item \textit{built with: }NodeJS, Express, Mongoose, JWT, bcrypt, amqplib, and Joi.
            \end{itemize}
        \end{itemize}\\

        2022  & \textbf{Notification Server}, [\href{https://github.com/zaky-fetoh/Learning-Backend-Development-Using-NodeJS/tree/main/Day10_NotificationSerWithCronJob}{Link}].\\&
        Periodically send a scheduled Notification using scheduled using CronJob. \\
    \end{longtable}
% \end{table}

% \end{small}

\end{document}